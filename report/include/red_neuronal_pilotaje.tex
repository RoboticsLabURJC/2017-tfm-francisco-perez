% CREATED BY DAVID FRISK, 2018
\chapter{Piloto visual con \textit{Deep Learning}}

\section{\textit{Dataset}}

\subsection{Aumento de datos}

\section{\textit{Drivers} JetBot}

\section{Redes de regresión}
\subsection{Arquitectura ResNet}
Contar la arquitectura resnet
\subsection{\textit{Transfer Learning} o \textcolor{blue}{Reentrenamiento}}
Contar la tecnica de reentrenamiento


\section{Validación experimental}

\subsection{Métricas de evaluación}
\subsection{Circuitos de test}
\subsection{Ejecución típica ResNet-34}
\subsection{Ejecución típica ResNet-18}
\subsection{Ejecución típica MobileNet}

\begin{comment}


\section{\textit{Drivers} del robot}

GAZEBO
* Conexión de behavior con el simulador Gazebo
* Piloto explícito para probar las funcionalidades de la aplicación (cambio de cerebro, cambio en caliente, cambio de entorno, grabación datasets, etc).
* Réplica del trabajo de vane, pruebas con los cerebros neuronales


ROBOT REAL
* Contar el desarrollo del driver controlador de los motores y cámara (con el envoltorio ROS).
* Contar la comunicación con la visualización? Lo dejamos en el modo headless?
* Contar las pruebas con el joystick


\section{Creación dataset}
* Creación manual
* Conjuntos de muestras obtenidos (250 vs 550)
* El conjunto pequeño, más variabilidad
* El conjunto grande menos variabilidad, más muestra
* El conjunto combinado, mantiene el sesgo de la variabilidad.
* Etiquetado para regresión. 
* Conversión de etiquetas.
* Normalización de la muestra
* \textcolor{blue}{Análisis de datos?}


\section{Piloto explícito???}
* Piloto basado en OpenCV para simulación


\section{Redes de regresión}
* Contexto de redes de regresión
* Redes residuales
* Arquitectura Resnet-18
* Arquitectura Resnet-34

\subsection{Métricas de evaluación}
* MSE
* MAE?

\subsection{Entrenamiento}
* Mencionar el cambio de versión de pytorch para hacerlo compatible con el pc de escritorio
* entrenado con una 970 (4 minutos) vs Jetson (1h30) (it/s?) con 250 muestras
* Gráficas de entrenamiento con diferentes redes para el loss (MSE)

\subsection{Rendimiento en diferentes pistas}
* Tiempos por vuelta en las pistas probadas



%% METER ESTA SECCION? O EXPLICARLA EN LA ANTERIOR??
\section{Aumento de datos}
* Denoising
* Horizontal random flips

\subsection{Rendimiento en diferentes pistas}
* No completa ninguna
\end{comment}